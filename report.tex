\documentclass[a4paper, 12pt]{article}

% Layout
\usepackage{geometry}
\geometry{left=16mm}
\geometry{right=10mm}
\geometry{top=1cm}
\geometry{bottom=1cm}

% Paragraph
\usepackage{indentfirst}
\setlength{\parindent}{0.75cm}
\linespread{1.25}

% Font
\usepackage{fontspec}
\usepackage[english,russian]{babel}
\usepackage{microtype}

% \usepackage{polyglossia}
% \setmainlanguage{russian}
% \setotherlanguage{english}

% \newfontfamily{\cyrillicfont}{Droid Serif}
% \newfontfamily{\cyrillicfontrm}{Droid Serif}
% \newfontfamily{\cyrillicfontsf}{Droid Sans}
% \newfontfamily{\cyrillicfonttt}{DejaVu Sans Mono}

\setmainfont{Times New Roman}
\setromanfont{Times New Roman}
\setsansfont{Droid Sans}
\setmonofont{DejaVu Sans Mono}

% Hyphens
\usepackage{hyphenat}
\usepackage{ucharclasses}
\setTransitionsForLatin{\begingroup\hyphenrules{english}}{\endgroup}

% Formulas
\usepackage{amssymb, amsfonts, amsmath}

% Miscellaneous
\usepackage{enumerate}
\usepackage{float}
\usepackage{multirow}

% Hyper references
\usepackage{hyperref}
\hypersetup{
    hidelinks,
    allcolors=black
}

% Images
\usepackage{graphicx}
\graphicspath{ {images/} }

%Including title
\usepackage{pdfpages}

% Figures
\usepackage{chngcntr}
\counterwithin{figure}{section}
\usepackage{subcaption}
\renewcommand\thesubfigure{\asbuk{subfigure})}
\captionsetup[subfigure]{labelformat=simple, labelsep=space}

% Counters
\usepackage[figure,table,page]{totalcount}
\usepackage{totcount}

% Code listings
\usepackage{listings}
\usepackage{xcolor}

\definecolor{codegreen}{rgb}{0,0.6,0}
\definecolor{codepurple}{rgb}{0.58,0,0.82}
\lstdefinestyle{codestyle}{
    commentstyle=\color{codegreen},
    keywordstyle=\color{magenta},
    stringstyle=\color{codepurple},
    basicstyle=\ttfamily\footnotesize,
    breakatwhitespace=false,
    breaklines=true,
    captionpos=b,
    keepspaces=true,
    showspaces=false,
    showstringspaces=false,
    showtabs=false,
    tabsize=2
}

\bibliographystyle{gost780s}



\newtotcounter{citenum} %From the package documentation
\def\oldbibitem{}
\let\oldbibitem=\bibitem
\def\bibitem{\stepcounter{citenum}\oldbibitem}

\begin{document}
\includepdf[pages={1}]
{title.pdf}

\tableofcontents
\newpage

\section{Введение}
Один из наиболее распространенных движущих факторов прогресса – искусственный
интеллект (далее – ИИ). Крайне широкий спектр его применения требует принятия
своевременных и эффективных мер по совершенствованию правового регулирования по
обширному кругу вопросов. Наиболее актуальные темы включают вопросы безопасности,
ответственности, использования и защиты данных, интеллектуальной собственности.

Цель данной работы -- обозначить общие правовые проблемы, возникающие при использовании ИИ,
ответственность, связанную с рисками применения самообучающихся алгоритмов для
решения различных задач, и привести примеры, иллюстрирующие эти задачи в реальной жизни.

\newpage
\section{Правосубъектность искусственного интеллекта}
Одной из главных мировых проблем является правосубъектность ИИ. Согласно \cite{chel},
проявление творческого потенциала ИИ не обеспечивает его включение как субъекта в рамки
права интеллектуальной собственности, основополагающие концепции которого в качестве автора,
способного к творческой деятельности, рассматривают исключительно физическое лицо. В статье
\cite{chel} данный вопрос, при взятых во внимание некоторых международных конференциях,
признается дискуссионным. Но в качестве наиболее вероятных решений проблемы приняты следующие положения:

\begin{enumerate}
\item Отказ от рассмотрения ИИ в качестве самостоятельного субъекта права интеллектуальной собственности.
\item Проведение разграничения результатов деятельности ИИ по критерию степени участия человека и
использования ИИ, а также установление различных правовых режимов для случаев, где ИИ играет
инструментальную роль, и случаев, где ИИ выступает с определенной степенью автономности.
\item Включение в национальное законодательство нормы о возможности использования законно
доступных объектов интеллектуальной собственности для целей обучения систем ИИ, если
правообладателем не выражено явное возражение против этого в надлежащей форме.
\end{enumerate}

В статье \cite{probs} автор придерживается аналогичного мнения. Он считает
обоснованным подход, в котором автором и правообладателем, соответственно,
будет признаваться владелец (собственник) робота. Неминуемо встаёт вопрос о том,
а каков будет правовой статус искусственного интеллекта после смерти его владельца?
Кто будет нести ответственность за все то, что успеет совершить ИИ? Автор считает, что
искусственный интеллект не может нести ответственность сам по себе, поскольку его действия
или бездействия зависят от оператора (владельца). Но этот вопрос открыт для обсуждения. И в
случае сильно развитого ИИ, законодатель обязан пересмотреть пределы ответственности самого
интеллекта, его оператора или создателя.
\newpage

\section{Риски при использовании ИИ и ответственность}
\newpage

\section{Примеры некоторых проблем}
\newpage

\section{Заключение}
\newpage

\begin{thebibliography}{}
\bibitem{chel}
Ю. О. Коряченкова, Искусственный интеллект: вызовы для права интеллектуальной
собственности, 2022

\bibitem{probs}
Сергеев А. В., Проблемы правовой охраны искусственного интеллекта в области
интеллектуальной собственности // Современная наука: актуальные проблемы
теории и практики, 2021. - С. 135-137

\bibitem{trans}
Коданева С.И., Трансформация интеллектуальной собственности под влиянием развития искусственного интеллекта.
(Обзор) // Социальные новации и социальные науки. – Москва : ИНИОН РАН, 2021. – № 2. – С. 132–141.

\bibitem{II}
Оморов Р. О., Интеллектуальная собственность и искусственный интеллект. // Технологии искусственного
интеллекта в менеджменте., 2020

\bibitem{civil}
А.А. Молчанов, Гражданско-правовые проблемы использования искусственного интеллекта в казённых учреждениях
системы МВД России в контексте права интеллектуальной собственности. // Вестник Санкт-Петербургского университета
МВД России, 2018. -№ 4 - C. 80

\bibitem{vac}
Шахназаров Б. А. Применение технологий искусственного интеллекта при создании вакцин и иных объектов интеллектуальной
собственности (правовые аспекты) // Актуальные проблемы российского права. — 2020. — Т. 15. — № 7. — С. 76—90

\bibitem{self}
Е. А. Войниканис, Е. В. Семенова, Г. С. Тюляев, Искусственный интеллект и право: вызовы и возможности
самообучающихся алгоритмов. // Вестник ВГУ. Серия: Право, 2018

\bibitem{abr}
Ролинсон П., Ариевич Е.А., Ермолина Д.Е., Объекты интеллектуальной собственности, создаваемые с помощью
искусственного интеллекта: особенности правового режима в России и за рубежом, 2018

\bibitem{reg}
Купчина Е. В., Искусственный интеллект и интеллектуальная собственность: вопросы правового регулирования патентных отношений
// Legal Concept = Правовая парадигма. – 2020. – Т. 19, № 4. С. 48–54.

\end{thebibliography}



\end{document}
